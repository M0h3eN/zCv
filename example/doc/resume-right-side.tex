\titleTwo[images/resume-pic.png]
\initcv
\makeprofile 
\section*{About myself}
An astronaut is a person who is trained and qualified to travel in space and perform various tasks while in orbit or on celestial bodies such as the Moon or planets. Astronauts are typically employed by government agencies such as NASA in the United States or the European Space Agency, or by private spaceflight companies. They are highly trained professionals who are responsible for operating spacecraft, conducting scientific experiments, and performing various other tasks while in space. In order to become an astronaut, individuals typically need to have a strong educational background in fields such as engineering, physics, or biology, as well as physical and mental stamina and the ability to handle the challenges and stresses of space travel. And yes the astronaut knows programming as well! :)

%----------------------------------------------------------------------------------------
%	 EDUCATION
%----------------------------------------------------------------------------------------

\section*{Education}

\begin{twenty} % Environment for a list with descriptions
	\twentyitem
	{2006 - 2008}
	{}
	{MSc. Astronomy\textnormal{(GPA: 4.0/4.0)}}
	{\href{http://www.astronomy.com/}{Yet another awesome university, Earth}}
	{}
	{}
	\twentyitem
	{2004-2006}
	{}
	{BS. Astronomy\textnormal{(GPA: 4.0/4.0)}}
	{\href{http://www.astronomy.com/}{The awesome university, Earth}}
	{}
	{}
	%\twentyitem{<dates>}{<title>}{<organization>}{<location>}{<description>}
\end{twenty}


\section*{Research}
\begin{twenty}
	\twentyitem
	{2019 - now}
	{}
	{Scientist}
	{\href{http://www.astronomy.com/}{Institute for Research in Astronamical science, Mars}}
	{}
	{
		{\begin{itemize}
				\item \textbf{Article}: A cool discovery about life on Mars!  
		\end{itemize}}
	} \\

	\twentyitem
	{2004 - 2006}
	{}
	{MSc. Candidate}
	{{Yet another awesome university}}
	{}
	{
		\textbf{Thesis}: Shuttle Training Aircraft!
	}
\end{twenty}


%----------------------------------------------------------------------------------------
%	 EXPERIENCE
%----------------------------------------------------------------------------------------

\section*{Experience}

\begin{twenty} % Environment for a list with descriptions
	\twentyitem
	{Dec 2019 - }
	{Now}
	{Astronaut}
	{\href{https://en.wikipedia.org/wiki/Mars}{Mars}}
	{}
	{\begin{itemize}
	\item Had the opportunity to see this fascinating planet up close and explore its surface

	\item Mars is a place of great interest to scientists and space enthusiasts alike, as it holds many clues about the history of our solar system and the possibility of life beyond Earth
	
	\item The surface of Mars is quite varied, with regions that include vast plains, towering mountains, and deep valleys
	
	\item The planet's thin atmosphere and extreme temperature fluctuations make it a challenging place to explore, but the potential scientific discoveries to be made there make it well worth the effort.
	
	\end{itemize}}
	\\
	\twentyitem
	{March 2014 - }
	{Dec 2019}
	{Astronaut}
	{\href{https://en.wikipedia.org/wiki/Jupiter}{Jupiter}}
	{}
	{\begin{itemize}
			\item The opportunity to see this massive planet up close, with its tumultuous clouds and giant red spot		
			\item The chance to explore the many moons of Jupiter, including Europa and Io
			\item Conducted scientific experiments and gathered data about Jupiter's atmosphere and moons
	\end{itemize}}
	\\
	
	\twentyitem
	{July 2012 - }
	{March 2014}
	{Astronaut}
	{\href{https://en.wikipedia.org/wiki/Saturn}{Saturn}}
	{}
	{\begin{itemize}
			\item The opportunity to see this beautiful gas giant up close, with its stunning rings and numerous moons
			\item The chance to explore the diverse moons of Saturn, including Titan and Enceladus
			\item Used the spacecraft's instruments to study the structure and composition of Saturn's atmosphere and rings
	\end{itemize}}
	\\
		\twentyitem
		{April 2010 -}
		{July 2012}
		{Astronaut}
		{\href{https://en.wikipedia.org/wiki/Moon}{Moon}}
		{}
		{
			{\begin{itemize}
				\item The opportunity to see Earth from a distance like no one else ever has
				\item The chance to explore the surface of the Moon and learn more about its geology and history
			\end{itemize}}
		}
		\\ 
\end{twenty}
